%%%%%%%%%%%%%%%%%
% This is an sample CV template created using altacv.cls
% (v1.3, 10 May 2020) written by LianTze Lim (liantze@gmail.com). Now compiles with pdfLaTeX, XeLaTeX and LuaLaTeX.
% (v1.6.5c, 27 Jun 2023) forked by Nicolás Omar González Passerino (nicolas.passerino@gmail.com)
%
%% It may be distributed and/or modified under the
%% conditions of the LaTeX Project Public License, either version 1.3
%% of this license or (at your option) any later version.
%% The latest version of this license is in
%%    http://www.latex-project.org/lppl.txt
%% and version 1.3 or later is part of all distributions of LaTeX
%% version 2003/12/01 or later.
%%%%%%%%%%%%%%%%

%% If you need to pass whatever options to xcolor
\PassOptionsToPackage{dvipsnames}{xcolor}

%% If you are using \orcid or academicons
%% icons, make sure you have the academicons
%% option here, and compile with XeLaTeX
%% or LuaLaTeX.
% \documentclass[10pt,a4paper,academicons]{altacv}

%% Use the "normalphoto" option if you want a normal photo instead of cropped to a circle
% \documentclass[10pt,a4paper,normalphoto]{altacv}

%% Fork (before v1.6.5a): CV dark mode toggle enabler to use a inverted color palette.
%% Use the "darkmode" option if you want a color palette used to 
% \documentclass[10pt,a4paper,ragged2e,withhyper,darkmode]{altacv}

\documentclass[10pt,a4paper,ragged2e,withhyper]{altacv}

%% AltaCV uses the fontawesome5 and academicons fonts
%% and packages.
%% See http://texdoc.net/pkg/fontawesome5 and http://texdoc.net/pkg/academicons for full list of symbols. You MUST compile with XeLaTeX or LuaLaTeX if you want to use academicons.

%% Fork v1.6.5c: Overwriting sloppy environment to ignore any spaces and be used to solve overlapping cvtags
\newenvironment{sloppypar*}{\sloppy\ignorespaces}{\par}

% Change the page layout if you need to
\geometry{left=1.2cm,right=1.2cm,top=1cm,bottom=1cm,columnsep=0.75cm}

% The paracol package lets you typeset columns of text in parallel
\usepackage{paracol}

% Change the font if you want to, depending on whether
% you're using pdflatex or xelatex/lualatex
\ifxetexorluatex
  % If using xelatex or lualatex:
  \setmainfont{Roboto Slab}
  \setsansfont{Lato}
  \renewcommand{\familydefault}{\sfdefault}
\else
  % If using pdflatex:
  \usepackage[rm]{roboto}
  \usepackage[defaultsans]{lato}
  % \usepackage{sourcesanspro}
  \renewcommand{\familydefault}{\sfdefault}
\fi

% Fork (before v1.6.5a): Change the color codes to test your personal variant on any mode
\ifdarkmode%
  \definecolor{PrimaryColor}{HTML}{C69749}
  \definecolor{SecondaryColor}{HTML}{D49B54}
  \definecolor{ThirdColor}{HTML}{1877E8}
  \definecolor{BodyColor}{HTML}{ABABAB}
  \definecolor{EmphasisColor}{HTML}{ABABAB}
  \definecolor{BackgroundColor}{HTML}{191919}
\else%
  \definecolor{PrimaryColor}{HTML}{001F5A}
  \definecolor{SecondaryColor}{HTML}{0039AC}
  \definecolor{ThirdColor}{HTML}{F3890B}
  \definecolor{BodyColor}{HTML}{666666}
  \definecolor{EmphasisColor}{HTML}{2E2E2E}
  \definecolor{BackgroundColor}{HTML}{E2E2E2}
\fi%

\colorlet{name}{PrimaryColor}
\colorlet{tagline}{SecondaryColor}
\colorlet{heading}{PrimaryColor}
\colorlet{headingrule}{ThirdColor}
\colorlet{subheading}{SecondaryColor}
\colorlet{accent}{SecondaryColor}
\colorlet{emphasis}{EmphasisColor}
\colorlet{body}{BodyColor}
\pagecolor{BackgroundColor}

% Change some fonts, if necessary
\renewcommand{\namefont}{\Huge\rmfamily\bfseries}
\renewcommand{\personalinfofont}{\small\bfseries}
\renewcommand{\cvsectionfont}{\LARGE\rmfamily\bfseries}
\renewcommand{\cvsubsectionfont}{\large\bfseries}

% Change the bullets for itemize and rating marker
% for \cvskill if you want to
\renewcommand{\itemmarker}{{\small\textbullet}}
\renewcommand{\ratingmarker}{\faCircle}

%% sample.bib contains your publications
%% \addbibresource{main.bib}

\begin{document}
    \name{Zoltán Szőcs}
    \tagline{Sr. Software Engineer}
    %% You can add multiple photos on the left or right
    % No photo for english CV
    \photoL{4cm}{../zoltan_portrait_sq}

    \personalinfo{
      \begin{minipage}[t]{0.3\textwidth}
        \email{zacsek@gmail.com}\smallskip
        \phone{+49-157-58094107}
        \location{Esslingen am Neckar, Germany}\\
        \linkedin{zoltan-szocs}
        \github{zacsek}
        %\npm{npmUser}
        %\dev{devtoUser}
        %\homepage{nicolasomar.me}
        %\medium{nicolasomar}
        %% You MUST add the academicons option to \documentclass, then compile with LuaLaTeX or XeLaTeX, if you want to use \orcid or other academicons commands.
        % \orcid{0000-0000-0000-0000}
        %% You can add your own arbtrary detail with
        %% \printinfo{symbol}{detail}[optional hyperlink prefix]
        % \printinfo{\faPaw}{Hey ho!}[https://example.com/]
        %% Or you can declare your own field with
        %% \NewInfoFiled{fieldname}{symbol}[optional hyperlink prefix] and use it:
        % \NewInfoField{gitlab}{\faGitlab}[https://gitlab.com/]
        % \gitlab{your_id}
      \end{minipage}
    }
      
    \makecvheader
    %% Depending on your tastes, you may want to make fonts of itemize environments slightly smaller
    % \AtBeginEnvironment{itemize}{\small}
    
    %% Set the left/right column width ratio to 6:4.
    \columnratio{0.25}

    % Start a 2-column paracol. Both the left and right columns will automatically
    % break across pages if things get too long.
    \begin{paracol}{2}
        % ----- TECH STACK -----
        
\cvsection{TECH STACK}
%% Fork v1.6.5c: The sloppypar* environment is used to avoid tags overlapping with section width
\begin{sloppypar*}
\cvtags{Ruby, Kotlin, Java, Ansible, Rails, PostgreSQL, MySQL, Linux, Docker, Git, RabbitMQ, Sidekiq}
\end{sloppypar*}
% ----- TECH STACK -----

% ----- LEARNING -----
\cvsection{Learning}
\begin{sloppypar}
\cvtags{Rust, ML/AI, AWS, Kubernetes, Terraform}
\end{sloppypar}

        % ----- LEARNING -----
        
        % ----- LANGUAGES -----
        \cvsection{Languages}
\cvlang{Hungarian}{Native}\\
    \cvlang{English}{Advanced / C1}
    \cvlang{German}{Advanced / C1}
    \cvlang{Romanian}{Advanced / C1}
 
            %% Yeah I didn't spend too much time making all the
            %% spacing consistent... sorry. Use \smallskip, \medskip,
            %% \bigskip, \vpsace etc to make ajustments.
        % ----- LANGUAGES -----
            
        % ----- REFERENCES -----
        %\cvsection{References}
        %    \cvref{Prof.\ Alpha Beta}{Institute}{a.beta@university.edu}
        %    \divider

        %    \cvref{Boss\ Gamma Delta}{Business}{g.delta@business.com}
        % ----- REFERENCES -----
        
        % ----- MOST PROUD -----
        % \cvsection{Most Proud of}
        
        % \cvachievement{\faTrophy}{Fantastic Achievement}{and some details about it}\\
        % \divider
        % \cvachievement{\faHeartbeat}{Another achievement}{more details about it of course}\\
        % \divider
        % \cvachievement{\faHeartbeat}{Another achievement}{more details about it of course}
        % ----- MOST PROUD -----
        
        % \cvsection{A Day of My Life}
        
        % Adapted from @Jake's answer from http://tex.stackexchange.com/a/82729/226
        % \wheelchart{outer radius}{inner radius}{
        % comma-separated list of value/text width/color/detail}
        % \wheelchart{1.5cm}{0.5cm}{%
        %   6/8em/accent!30/{Sleep,\\beautiful sleep},
        %   3/8em/accent!40/Hopeful novelist by night,
        %   8/8em/accent!60/Daytime job,
        %   2/10em/accent/Sports and relaxation,
        %   5/6em/accent!20/Spending time with family
        % }
        
        % use ONLY \newpage if you want to force a page break for
        % ONLY the current column
        \newpage
        
        %% Switch to the right column. This will now automatically move to the second
        %% page if the content is too long.
        \switchcolumn
        
        % ----- ABOUT ME -----
        \cvsection{About Me}
            \begin{quote}
With over 20 years of experience in software development, including more than a decade dedicated to Ruby, I've honed a deep expertise in crafting efficient, scalable solutions. 
My journey, enriched by a versatile background in C++ and Java, is marked by significant achievements, such as leading the development of a crucial verification tool for Daimler, impacting millions of vehicles. 
Known for my analytical approach and rapid adaptability, I take pride in creating high-throughput software and fostering team member growth. 
As I aspire towards roles in tech leadership and software architecture, my goal is to continue developing innovative Ruby-based projects, pushing the boundaries of technology and teamwork.   
\end{quote}


        % ----- ABOUT ME -----
        
        % ----- EXPERIENCE -----
        \cvsection{Experience}
            \cvevent{Sr. Software Engineer}{Wolters Kluwer}{2021 -- current}{Ludwigsburg, Germany}
\begin{itemize}
  \item Designed and implemented a novel mechanism for generating payment advice documents, creating detailed PDF summaries of outstanding payments.
  \item Reengineered and enhanced workflows for receipt handling and payment generation within the SMART Connect cloud-based application.
\end{itemize}
\divider

            
            \cvevent{Tech Lead}{Daimler extern / KPIT}{2015 -- 2020}{Stuttgart, Germany}{Ruby, PostgreSQL, Sidekiq}
\begin{minipage}{\linewidth}
\begin{itemize}
  \item Spearheaded the development of an internal verification tool within Daimler's Truck department, implemented globally across approximately 5 million Mercedes trucks.
  \item Created and optimized a production environment using JRuby / Sidekiq, tailored to client specifications, handling over 500 GB of structured data daily.
  \item Guided a team of 12 based in India in developing cloud-based fleet management software for a German client, ensuring effective international collaboration and project alignment.
\end{itemize}
\end{minipage}
\divider


            \cvevent{Quantitative Developer}{Citi}{2014 -- 2015}{Budapest, Hungary}
\begin{itemize}
  \item Engineered a bond rate spider that aggregated data from diverse sources, processed it, and distributed it globally, enabling real-time queries from within Excel sheets.
  \item Managed and optimized a global server pool spanning 6 locations with over 100 servers, ensuring high efficiency and reliability.
  \item Implemented financial algorithms in C++ from Quantitative Analysts' proof-of-concept papers, translating theoretical models into practical applications.
\end{itemize}
\divider


            \cvevent{Sr. Software Engineer, Android}{LogMeIn (Developer of Remote Access and Support Solutions)}{2011 -- 2014}{Budapest, Hungary}{Java, Android, Ruby}
\begin{minipage}{\linewidth}
\begin{itemize}
  \item Pioneered the development of a screen capture mechanism for LogMeIn's remote assistance Android app, marking (possibly) the world's first such commercial implementation of this technology at the time.
  \item Transformed the Rescue app from a basic proof-of-concept to a globally deployed business application.
  \item Collaborated with OEM partners (Samsung, HTC, Sony-Ericsson, LG) in the software integration of the application.
\end{itemize}
\end{minipage}
\divider


            \newpage

            \cvevent{Sr. Software Engineer}{Intellio (Creator of Intelligent IP Security Cameras)}{2007 -- 2011}{Budapest, Hungary}
\begin{itemize}
  \item Developed a Linux environment for a network-attached intelligent surveillance camera, including integrating hardware support in the Linux kernel.
  \item Engineered a comprehensive framework for the embedded application on the camera, encompassing server communication and streaming protocols.
  \item Implemented bounding box-based movement tracking and alerting features, enhancing real-time event recognition and reporting capabilities.
\end{itemize}
\divider


            \cvevent{Software Developer}{Nokia Siemens Networks}{2006 -- 2007}{Budapest, Hungary}
\begin{itemize}
  \item Developed account authentication features for Nokia's 3G IMS solution.
  \item Written automated tests for functional features in the 3G and legacy 2G solutions.
\end{itemize}
\divider


            \cvevent{Software Developer}{Neogen}{2004 -- 2006}{Targu Mures, Romania}{PHP, C++, MySQL}
\begin{minipage}{\linewidth}
\begin{itemize}
  \item Conceived and developed a search engine, used in ad-serving processes for affiliate websites. 
  \item Engineered advanced statistics generation features for the company’s intranet, enhancing data analysis and reporting capabilities.
  \item Developed the feature for recognizing, validating, and importing e-invoices in SMART Connect.
\end{itemize}
\end{minipage}
\divider

        % ----- EXPERIENCE -----
        
        % ----- EDUCATION -----
        \cvsection{Education}
            \cvevent{Master of Science (MSc) in Automation Engineering}{Petru Maior University}{2000 -- 2005}{Targu Mures, Romania}
\divider

            
        % ----- EDUCATION -----
        
        % ----- PROJECTS -----
        %\cvsection{Projects}
        %    \cvevent{Project 1 }{\cvreference{\faGithub}{https://github.com/user/repo}\cvreference{| \faGlobe}{https://project-demo.com/}}{Mm YYYY -- Mm YYYY}{}
        %    \begin{itemize}
        %        \item Item 1
        %        \item Item 2
        %    \end{itemize}
        %    \divider
        %    
        %    \cvevent{Project 2 }{\cvreference{\faGitlab}{https://gitlab.com/user/repo}\cvreference{| \faGlobe}{https://project-demo.com/}}{Mm YYYY -- Mm YYYY}{}
        %    \begin{itemize}
        %        \item Item 1
        %        \item Item 2
        %    \end{itemize}
        % ----- PROJECTS -----
    \end{paracol}
\end{document}
